%==============================================================================
% Sjabloon onderzoeksvoorstel bachproef
%==============================================================================
% Gebaseerd op document class `hogent-article'
% zie <https://github.com/HoGentTIN/latex-hogent-article>

% Voor een voorstel in het Engels: voeg de documentclass-optie [english] toe.
% Let op: kan enkel na toestemming van de bachelorproefcoördinator!
\documentclass{hogent-article}

% Invoegen bibliografiebestand
\addbibresource{voorstel.bib}
\usepackage{float}
% Informatie over de opleiding, het vak en soort opdracht
\studyprogramme{Professionele bachelor toegepaste informatica}
\course{Bachelorproef}
\assignmenttype{Onderzoeksvoorstel}
% Voor een voorstel in het Engels, haal de volgende 3 regels uit commentaar
% \studyprogramme{Bachelor of applied information technology}
% \course{Bachelor thesis}
% \assignmenttype{Research proposal}

\academicyear{2025-2026} % TODO: pas het academiejaar aan

% TODO: Werktitel
\title{AI-gedreven anomaliedetectie in web- en API-verkeer: ontwikkeling van een proof-of-concept binnen de securityomgeving van Evolane}

% TODO: Studentnaam en emailadres invullen
\author{Phil Van Akoleyen}
\email{phil.vanakoleyen@student.hogent.be}

\projectrepo{https://github.com/philvanakoleyen/Bachelorproef_PhilVanAkoleyen.git}

% TODO: Medestudent
% Gaat het om een bachelorproef in samenwerking met een student in een andere
% opleiding? Geef dan de naam en emailadres hier
% \author{Yasmine Alaoui (naam opleiding)}
% \email{yasmine.alaoui@student.hogent.be}

% TODO: Geef de co-promotor op
\supervisor[Co-promotor]{}

% Binnen welke specialisatierichting uit 3TI situeert dit onderzoek zich?
% Kies uit deze lijst:
%
% - Mobile \& Enterprise development
% - AI \& Data Engineering
% - Functional \& Business Analysis
% - System \& Network Administrator
% - Mainframe Expert
% - Als het onderzoek niet past binnen een van deze domeinen specifieer je deze
%   zelf
%
\specialisation{System \& Network Administrator}
\keywords{API-Security, anomalie detectie, Machine Learning}

\begin{document}

\begin{abstract}
  De frequentie en de complexiteit van cyberaanvallen op webapplicaties en API’s nemen steeds meer toe, met tot gevolg dat het voor security operations teams steeds moeilijker wordt om afwijkend gedrag tijdig te detecteren met klassieke rule-based detectiesystemen. Binnen een bedrijf als Evolane, waar web- en API-verkeer wordt gemonitord en beveiligd via Akamai App en API protector in combinatie met het een Security Information and Event Management platform, leidt dit tot een verhoogde werkdruk en een toename in false positives. Deze bachelorproef wil daarom onderzoeken hoe een AI-gedreven anomalie detectiesysteem op basis van Akamai access logs en Security Information and Event Management data automatisch afwijkend gedrag in web- en API-verkeer kan detecteren en false positives kan reduceren binnen Evolane’s securityomgeving. Het doel van deze bachelorproef is het ontwikkelen van een proof-of-concept dat anomalieën identificeert op basis van afwijkingen in het gedrag en de false positive rate weet te verlagen. Het onderzoek start bij het uitvoeren van een literatuurstudie rond anomaliedetectie en machine learning in cybersecurity, gevolgd door een requirementsanalyse via overleg met stakeholders binnen Evolane. Hierna wordt een dataset opgebouwd uit Akamai logs en SIEM-data deze wordt genormaliseerd en doormiddel van feature engineering worden relevante kenmerken eruit gehaald, waarna er een machine learning gedreven anomalie detectie methode op de data gebruikt wordt. De evaluatie focust zich op de detectieperformantie, vermindering van false positives en de operationele bruikbaarheid. Er wordt verwacht dat het prototype in staat zal zijn om afwijkend gedrag te signaleren en een meerwaarde zal zijn voor de omgeving van Evolane en die van hun klanten.
\end{abstract}

\tableofcontents

% De hoofdtekst van het voorstel zit in een apart bestand, zodat het makkelijk
% kan opgenomen worden in de bijlagen van de bachelorproef zelf.
%---------- Inleiding ---------------------------------------------------------

% TODO: Is dit voorstel gebaseerd op een paper van Research Methods die je
% vorig jaar hebt ingediend? Heb je daarbij eventueel samengewerkt met een
% andere student?
% Zo ja, haal dan de tekst hieronder uit commentaar en pas aan.

%\paragraph{Opmerking}

% Dit voorstel is gebaseerd op het onderzoeksvoorstel dat werd geschreven in het
% kader van het vak Research Methods dat ik (vorig/dit) academiejaar heb
% uitgewerkt (met medesturent VOORNAAM NAAM als mede-auteur).
% 

\section{Inleiding}%
\label{sec:inleiding}

De afgelopen jaren is de complexiteit en de frequentie van aanvallen op webapplicaties en Application Programming Interfaces (API's) sterk toegenomen. Binnen Evolane wordt web- en API-verkeer gemonitord en beveiligd door Akamai App & API Protector en een Security Information and Event Management platform. Doordat cyberaanvallen steeds complexer worden en eveneens gebruik maken van automatisatie wordt het steeds moeilijker om afwijkend gedrag in web- en API-verkeer tijdig te detecteren met klassieke rule-based detectiemechanismen. Deze aanpak leidt tot een verhoogde werkdruk en een toename in het aantal false positives binnen het Security Operations Team. Om deze problematiek aan te pakken ontstaat er nood aan een aanvullend detectiesysteem dat in staat is om afwijkend gedrag automatisch te herkennen op basis van gedragsafwijkingen in grote hoeveelheden log- en event data. Deze bachelorproef richt zich op het ontwikkelen van een proof-of-concept dat Evolane ondersteunt bij het verminderen van false positives en het verbeteren van de detectieperformantie. De centrale onderzoeksvraag luidt daarom: Hoe kan een machine-learninggedreven anomaliedetectiesysteem op basis van Akamai-accesslogs en SIEM-data worden ontworpen en geïmplementeerd om binnen de securityomgeving van Evolane afwijkend web- en API-verkeer te detecteren en het aantal false positives te reduceren? Om deze vraag te beantwoorden wordt eerst het probleemdomein onderzocht. Welke types aanvallen en afwijkend gedrag komen het vaakst voor in het web en API verkeer dat Evolane monitort? En welke functionele en niet functionele vereisten stelt Evolane aan een aanvullend detectiesysteem? Vervolgens wordt nagegaan welke machine-learningbenaderingen (supervised, unsupervised of semi-supervised) het meest geschikt zijn voor anomaliedetectie binnen het Akamai en SIEM datadomein van Evolane, en welke features uit deze datasets het meest indicatief zijn voor afwijkend of verdacht gedrag? Tot slot wordt er geëvalueerd in welke mate het prototype in staat is om false positives te reduceren ten opzichte van traditionele rule-based detectiesystemen en bruikbare detecties kan opleveren in een operationele SOC-context. 

%---------- Stand van zaken ---------------------------------------------------

\section{Literatuurstudie}%
\label{sec:literatuurstudie}

\subsection{Web- en API-beveiliging: context en uitdagingen}

De afgelopen jaren is de complexiteit van web en API-aanvallen sterk toegenomen. De OWASP API security top 10 benadrukt dat API aanvallen vaak het gevolg zijn van gebrek aan autorisatie, onvoldoende controle op resourcegebruik en misbruik van businesslogica, wat het detecteren van een aanval aanzienlijk complexer maakt \textcite{OWASP2023} Rapporten van Akamai tonen aan dat er in 2024 een toename is van 33 procent in aanvallen gericht op web applicaties en API. Deze stijging wordt in verband gebracht met de toename van cloudgebaseerde architecturen, microservices en AI gedreven applicaties.
Daarnaast wordt er een sterke automatisering van aanvallen waargenomen, waar bots gedreven door AI en geautomatiseerde scripts worden ingezet voor onder andere denial of service, credential stuffing en API misbruik \autocite{Akamai2025}. Dit maakt het ontzettend moeilijk voor deze aanvallen te onderscheiden van normaal API verkeer aangezien deze aanvallen vaak gebruikmaken van geldige requests die zich kunnen aanpassen aan de beveiligingsmaatregelen. Het gevolg hiervan is dat klassieke rulebased beveiligingsmechanismen, die steunen op vooraf gedefinieerde signatures en patronen steeds meer beperkingen ondervinden in het detecteren van nieuwe dreigingen \autocite{Standards2007}.


\subsection{Anomaly detection en threat intelligence binnen Security Operations}

NIST wijst erop dat organisaties vaak geconfronteerd worden met grote hoeveelheden heterogene logdata en beperkte middelen voor analyse, wat het correleren en interpreteren van security-events bemoeilijkt binnen een SOC-context \Autocite{Kent2006}. In de omgeving van Evolane wordt Web en API-verkeer reeds actief beschermt en geanalyseerd door Akamai APP en API protector in combinatie met een SIEM, hoewel deze oplossingen effectieve detectie- en preventiemechanismen bieden is er binnen de context van het security operations team nog steeds nood aan een aanvullende analyse toepassing die abnormaal gedrag detecteert, false positives vermindert en real time correlaties kan leggen tussen security events. In literatuur wordt de fusie van machine-learning gedreven anomaliedetectie met traditionelere intrusion detection systemen beschreven als een waardevolle aanvulling, omdat dergelijke systemen die gebruikmaken van machine learning grote hoeveelheden netwerk en log data kunnen analyseren terwijl de afhankelijkheid van vastgestelde regels en signatures wordt verminderd \Autocite{Kumar_2025}.

\subsection{Machine-learningmethodes voor anomaliedetectie in web- en API-verkeer}

Machine-learning anomaliedetectiesystemen worden in de literatuur onderverdeeld in supervised, unsupervised en semi-supervised benaderingen. Deze indeling en bijhorende eigenschappen zijn gebaseerd op de overzichtsstudie van Saabith en Vinothraj \Autocite{Saabith2023}.
Supervised learning vereist gelabelde datasets waar expliciet normaal en afwijkend gedrag is aangeduid. Tijdens de training leert het model de relatie tussen de invoerkenmerken en de geassocieerde labels, waardoor het instaat is om gekende anomalieën te identificeren. Hoewel deze benadering uitermate geschikt is in het detecteren van gekende anomalieën vanuit zijn dataset, is de toepasbaarheid eerder beperkt door het gebrek aan betrouwbare en representatieve gelabelde datasets.
Unsupervised learning benaderingen leert Anomalieën herkennen gebaseerd op patronen en afwijkend gedrag in de data, zonder nood te hebben aan gelabeld trainingsmateriaal. Anomalieën worden beschouwd als datapunten die afwijken van het dominante gedrag of niet behoren tot een bestaande cluster. Deze algoritmen voegen gegevenspunten samen op basis van overeenkomsten en verschillen. Deze methoden zijn geschikt voor het identificeren van ongekende Anomalieën en het identificeren van complexe patronen in data, maar identificeert meer false positives door gebrek aan labeld data en kan mogelijk geen onderscheid maken tussen verschillende Anomalieën.
Semisupervised learning vormt een tussenvorm hierbij wordt het model voornamelijk getrained op niet-gelabelde data, aangevuld met een beperkte hoeveelheid gelabelde gegevens of vooraf gedefinieerde profielen van normaal gedrag. Niet gelabelde data wordt gebruikt om de accuraatheid van het model te verbeteren, terwijl gelabelde data ervoor zorgt dat het algoritme directer anomalieën kan detecteren. Deze methode zorgt ervoor dat de voordelen van unsupervised detectie worden gecombineerd met extra context van labeld data, waardoor de detectie van anomalieën wordt verfijnt en het aantal false positives worden gereduceerd \Autocite{Saabith2023}.


\subsection{Securitydata en feature-extractie voor anomaliedetectie }

Anomaliedetectie binnen Web- en API-beveiliging is strek afhankelijk van de beschikbare data, slechte data kwaliteit zoals onvolledige, inconsistente API metrics zorgen voor onnauwkeurige resultaten en verhogen het risico op false positives en negatives. Het verzekeren van data integriteit door correcte collectie, opslag en preprocessing is cruciaal voor effectieve anomalie detectie \Autocite{Meegle2025}.
Tegenovergesteld aan klassieke netwerkaanvallen bestaat misbruik bij web- en API-omgevingen vaak uit afwijkend gebruik van legitieme functionaliteit, dit zorgt ervoor dat anomalieën zich eerder zichtbaar maken op gedragsniveau in plaats van via duidelijke signaturen \Autocite{OWASP2023}
Ruwe logdata kan niet rechtstreeks gebruikt worden voor machine-learning gebaseerde anomaliedetectie. Daarom is feature-extractie een essentiële stap, waarbij de parameters gebruikt worden als data punten in een dataset zodat afwijkingen ten opzichte van normaal gedrag kunnen worden geïdentificeerd Veelgebruikte features zijn onder andere: 

\begin{itemize}
    \item Client IP addres
    \item API endpoint 
    \item Response Data Size
    \item Status code 
    \item Timestamp
    \item Request type
\end{itemize}

Onderzoek naar API traffic anomaly detection toont aan dat dergelijke feature gebaseerde representaties geschikt zijn voor machine learning unsupervised of semi supervised anomaliedetectie waar data niet betrouwbaar gelabeld is \Autocite{Sowmya2023}.


\subsection{Evaluatiecriteria en meerwaarde van anomaliedetectie in een SOC-context}

De keuze in Evaluatiecriteria is afhankelijk van toepassingscontext als de kenmerken van de beschikbare data. Meerdere evaluatiemaatstaven worden gebruikt voor het analyseren van de performantie en meerwaarde van een machine-learning gebasseerd algoritme in anomalie detectie\Autocite{Saabith2023}. 

%---------- Methodologie ------------------------------------------------------
\section{Methodologie}%
\label{sec:methodologie}

De bachelorproef start met een literatuurstudie die zal dienen als theoretische basis voor het onderzoek rond anomaliedetectie in web- en API-verkeer. Gelijklopend hiermee wordt er een analyse gemaakt van de context van Evolane ook wordt er een requirement analyse gemaakt die zal dienen als referentiekader, waarna in opeenvolgende fasen een dataset wordt opgebouwd, een Proof of Concept ontwikkeld en de resultaten geëvalueerd. Doorheen het onderzoek zullen alle fasen van het onderzoek worden gedocumenteerd. 

\subsection{Literatuurstudie}

Om het onderzoeksprobleem correct te kunnen situeren, wordt er een literatuurstudie uitgevoerd rond anomaliedetectie, machine learning in cybersecurity en loganalyse. Deze literatuurstudie heeft als doel: 
- Inzicht verkrijgen in moderne technieken voor detectie van afwijkend gedrag in web- en API-verkeer. 
- Bestaande methoden en modellen te identificeren die relevant zijn voor de opbouw van het prototype. 
- De noodzaak en toegevoegde waarde van een ML-gedreven benadering binnen de context van Evolane te onderbouwen
De bevindingen in de literatuurstudie vormen de basis voor de requirement analyse en het bepalen welke technieken in het Proof of Concept onderzocht worden. 


\subsection{Requirementsanalyse }

Gelijklopend met de literatuurstudie wordt er een requirement analyse uitgevoerd om het onderzoeksprobleem scherp af te bakenen, dit gebeurt aan de hand van interviews met belanghebbenden. De verzamelde vereisten worden in overleg met Evolane geprioriteerd. 

\subsection{Dataverzameling en normalisatie}

Om een reproduceerbare en herbruikbare dataset op te bouwen wordt een datastroom opgezet bestaande uit Akamai access logs en SIEM-events. De verzamelde gegevens worden genormaliseerd naar een homogene dataset. Python wordt gebruikt voor data cleaning, parsing en preprocessing te automatiseren. Het resultaat is een gestructureerde dataset in CSV-formaat.

\subsection{Feature engineering en gedragsmodellering}

Uit de genormaliseerde data worden betekenisvolle kenmerken geëxtraheerd die gebruikt kunnen worden voor het detecteren van afwijkend gedrag in web- en API-verkeer. De selectie van deze kenmerken gebeurt op basis van inzichten uit de literatuurstudie. De data wordt samengevoegd over tijdsvensters en per endpoint of client, zodat we gedragsprofielen kunnen opbouwen. Het resultaat van deze fase is een feature-set die geschikt is voor toepassing binnen het Proof of Concept. 

\subsection{Proof-of-Concept}

Om de theoretische inzichten te toetsen wordt er een Proof of Concept uitgewerkt waarin een anomaliedetectiemethode wordt geïmplementeerd. De implementatie van dit machine learning model zal worden getest en gevalideerd. Het uiteindelijke resultaat kan dan gebruikt worden om te demonstreren aan de belanghebbenden hoe dit systeem zou werken binnen de werkelijke infrastructuur van Evolane. 

\subsection{Documentatie}

Doorheen het gehele proces van de bachelorproef wordt er stapsgewijs documentatie opgesteld voor iedere fase. Hier kan later terug naar verwezen worden bij vragen, en kan ook gebruikt worden voor de implementatie te reproduceren. 

\subsection{Conclusie & advies}

Eenmaal alles is opgesteld en de bijhorende evaluatie is uitgevoerd wordt er een conclusie geformuleerd op basis van de bekomen resultaten. Hierbij wordt nagegaan in welke mate een machine-learning gedreven aanpak een meerwaarde biedt ten opzicht van klassieke detectie mechanismen. Op basis van de analyse wordt een onderbouwd advies geformuleerd meet aandacht voor toepasbaarheid, voordelen, beperkingen en mogelijke vervolgstappen.

\FloatBarrier
\subsection{Planning}

\begin{figure}[H]
    \centering
    \includegraphics[width=\columnwidth]{Gantt_Diagram.png}
    \caption{Gantt diagram met de verschillende fasen en milestones van het onderzoek.}
    \label{fig:gantt}
\end{figure}

%---------- Verwachte resultaten ----------------------------------------------
\section{Verwacht resultaat, conclusie}%
\label{sec:verwachte_resultaten}

Het verwachte resultaat van dit onderzoek is een werkend proof of concept waarin afwijkend gedrag gedetecteerd kan worden op basis van Akamai-acceslogs en SIEM-data. Het PoC zal aantonen dat machine learning gedreven methoden geschikt zijn voor het detecteren van nieuwe anomalieen, daarnaast zal het ook een vermindering aantonen in het aantal false positives. Deze resultaten zullen dan een basis vormen voor het antwoord op de hoofd- en deelvragen.

De Proof of Concept zal een duidelijk beeld geven aan Evolane wat er precies mogelijk is met een machine learning gedreven anomalie detectiesyteem. Verder zullen de fasen van het onderzoek aantonen wat de voordelen en nadelen zijn. Tot slot zal de verwachte conclusie zijn dat een ML-gedreven anomalie detectiesysteem technisch haalbaar is in de omgeving van Evolane en een voordeel kan zijn voor de beveiliging van de omgevingen van hun klanten.



\printbibliography[heading=bibintoc]

\end{document}